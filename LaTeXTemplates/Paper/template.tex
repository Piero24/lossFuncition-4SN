%%%%%%%%%%%%%%%%%%%%%%%%%%%%%%%%%%%%%%%%%
% Journal Article
% LaTeX Template
% Version 2.0 (February 7, 2023)
%
% This template originates from:
% https://www.LaTeXTemplates.com
%
% Author:
% Vel (vel@latextemplates.com)
%
% License:
% CC BY-NC-SA 4.0 (https://creativecommons.org/licenses/by-nc-sa/4.0/)
%
% NOTE: The bibliography needs to be compiled using the biber engine.
%
%%%%%%%%%%%%%%%%%%%%%%%%%%%%%%%%%%%%%%%%%

%----------------------------------------------------------------------------------------
%	PACKAGES AND OTHER DOCUMENT CONFIGURATIONS
%----------------------------------------------------------------------------------------

\documentclass[
	a4paper, % Paper size, use either a4paper or letterpaper
	10pt, % Default font size, can also use 11pt or 12pt, although this is not recommended
	unnumberedsections, % Comment to enable section numbering
	twoside, % Two side traditional mode where headers and footers change between odd and even pages, comment this option to make them fixed
]{LTJournalArticle}

\addbibresource{sample.bib} % BibLaTeX bibliography file

\runninghead{Shortened Running Article Title} % A shortened article title to appear in the running head, leave this command empty for no running head

\footertext{\textit{Deep Learning course} (2024) 12:533-684} % Text to appear in the footer, leave this command empty for no footer text

\setcounter{page}{1} % The page number of the first page, set this to a higher number if the article is to be part of an issue or larger work

%----------------------------------------------------------------------------------------
%	TITLE SECTION
%----------------------------------------------------------------------------------------

\title{HSNet+: Enhancing Polyp Segmentation with Region-wise Loss} % Article title, use manual lines breaks (\\) to beautify the layout

% Authors are listed in a comma-separated list with superscript numbers indicating affiliations
% \thanks{} is used for any text that should be placed in a footnote on the first page, such as the corresponding author's email, journal acceptance dates, a copyright/license notice, keywords, etc
\author{Pietrobon Andrea and Biiffis Nicola}


% Affiliations are output in the \date{} command
\date{Department of Engineering Information, University of Padua}


% Full-width abstract
\renewcommand{\maketitlehookd}{%
	\begin{abstract}
		\noindent In this research we present a novel approach for polyp segmentation by integrating loss per region (RW) into the hybrid semantic network (HSNet) architecture.RW loss is a versatile and effective technique that addresses the class imbalance and importance of pixels. Implementing RW loss as a multiplication in pixels between the softmax output and a RW map.We investigate the optimization instability associated with specific RW maps, especially boundary loss distance maps. This principle ensures convergence without the need for additional regularization terms, making it adaptable to various datasets. As a result, we use a simplified version of boundary distance maps called region-adjusted maps (RRW), which demonstrate outstanding stability and achieve state-of-the-art performance in various segmentation tasks. In this study, we apply the RW loss integration approach within the HSNet architecture, which combines the strengths of Transformer and convolutional neural networks (CNNs). \\
Several experiments were conducted to evaluate our approach against different cutting edge on five benchmark datasets. The results demonstrate the superior performance of our proposed approach, achieving higher accuracy across various evaluation metrics.
	\end{abstract}
}

%----------------------------------------------------------------------------------------

\begin{document}

\maketitle % Output the title section

%----------------------------------------------------------------------------------------
%	ARTICLE CONTENTS
%----------------------------------------------------------------------------------------

\section{Introduction}

In the field of biomedicine, quantitative analysis requires a crucial step: image segmentation. Manual segmentation is a time-consuming and subjective process, as demonstrated by the considerable discrepancy between segmentations performed by different annotators (as highlighted in [R2]). Consequently, there is a strong interest in developing reliable tools for automatic segmentation of medical images [R3].

The use of neural networks to automate polyp segmentation can provide physicians with an effective tool for identifying such formations or areas of interest during clinical practice. However, there are two important challenges that limit the effectiveness of this segmentation:

\begin{enumerate}
	\item Polyps can vary significantly in size, orientation, and illumination, making accurate segmentation difficult to achieve.
	\item Current approaches often overlook significant details such as textures.
\end{enumerate}

To obtain precise segmentations in medical image segmentation, it is crucial to consider class imbalance and the importance of individual pixels. By pixel importance, we refer to the phenomenon where the severity of classification errors depends on the position of such errors.

Current approaches for polyp segmentation primarily rely on convolutional neural networks (CNN) or Transformers. To overcome the mentioned challenges, this research proposes the use of a hybrid semantic network (HSNet) that combines the advantages of Transformer networks and convolutional neural networks (CNN), along with regional loss (RW). This global approach simultaneously takes into account class imbalance and pixel importance, without requiring additional hyperparameters or loss functions, in order to improve polyp segmentation.

HSNet includes a Cross-Semantic Attention (CSA) module that bridges the gap between low-level and high-level features through an interactive mechanism that exchanges two types of semantics derived from different neural network attentions. Through a dual-branch structure of Transformer and CNN, the Hybrid Semantic Complement (HSC) module is designed, which captures both long-range dependencies and local details of the appearance. The Multi-Scale Prediction (MSP) module learns weights for fusing stage-level prediction masks in a decoder.

Boundary loss [R13] and Hausdorff loss (HD) [R12] are difficult to optimize, and as explained in [12, 13], they need to be combined with Dice loss to overcome this optimization instability. However, it is not clear whether Dice loss specifically needs to be paired with boundary or HD losses to achieve stable optimization.

Furthermore, the precise cause of this optimization instability is still unknown. According to Kervadec et al. [13], optimizing distance maps from the boundary occasionally fails because empty foregrounds (all zeros in the foreground class) produce minimal gradients. This hypothesis describes a typical class imbalance scenario where most pixels outweigh foreground pixels, and consequently, the few foreground pixels contribute only marginally to the gradients.

In this study, we examine how the implementation of regional loss (RW) affects polyp segmentation by applying it to a hybrid semantic network (HSNet).


%------------------------------------------------

\section{Methodologies}

\subsection{Sample Sites \& Processing}

This line shows how to use a footnote to further explain or cite text\footnote{Example footnote text.}.

This is a bullet point list:

\begin{itemize}
	\item Arcu eros accumsan lorem, at posuere mi diam sit amet tortor
	\item Fusce fermentum, mi sit amet euismod rutrum
	\item Sem lorem molestie diam, iaculis aliquet sapien tortor non nisi
	\item Pellentesque bibendum pretium aliquet
\end{itemize}

Mauris interdum porttitor fringilla. Proin tincidunt sodales leo at ornare. Donec tempus magna non mauris gravida luctus. Cras vitae arcu vitae mauris eleifend scelerisque. Nam sem sapien, vulputate nec felis eu, blandit convallis risus. Pellentesque sollicitudin venenatis tincidunt. In et ipsum libero. Nullam tempor ligula a massa convallis pellentesque.

This is a numbered list:

\begin{enumerate}
	\item Donec dolor arcu, rutrum id molestie in, viverra sed diam
	\item Curabitur feugiat
	\item Turpis sed auctor facilisis
\end{enumerate}

\subsection{Species Identification}

Proin lobortis efficitur dictum. Pellentesque vitae pharetra eros, quis dignissim magna. Sed tellus leo, semper non vestibulum vel, tincidunt eu mi. Aenean pretium ut velit sed facilisis. Ut placerat urna facilisis dolor suscipit vehicula. Ut ut auctor nunc. Nulla non massa eros. Proin rhoncus arcu odio, eu lobortis metus sollicitudin eu. Duis maximus ex dui, id bibendum diam dignissim id. Aliquam quis lorem lorem. Phasellus sagittis aliquet dolor, vulputate cursus dolor convallis vel. Suspendisse eu tellus feugiat, bibendum lectus quis, fermentum nunc. Nunc euismod condimentum magna nec bibendum. Curabitur elementum nibh eu sem cursus, eu aliquam leo rutrum. Sed bibendum augue sit amet pharetra ullamcorper. Aenean congue sit amet tortor vitae feugiat.

Mauris interdum porttitor fringilla. Proin tincidunt sodales leo at ornare. Donec tempus magna non mauris gravida luctus. Cras vitae arcu vitae mauris eleifend scelerisque. Nam sem sapien, vulputate nec felis eu, blandit convallis risus. Pellentesque sollicitudin venenatis tincidunt. In et ipsum libero. Nullam tempor ligula a massa convallis pellentesque.

\subsection{Data Analysis}

Vestibulum sodales orci a nisi interdum tristique. In dictum vehicula dui, eget bibendum purus elementum eu. Pellentesque lobortis mattis mauris, non feugiat dolor vulputate a. Cras porttitor dapibus lacus at pulvinar. Praesent eu nunc et libero porttitor malesuada tempus quis massa. Aenean cursus ipsum a velit ultricies sagittis. Sed non leo ullamcorper, suscipit massa ut, pulvinar erat. Aliquam erat volutpat. Nulla non lacus vitae mi placerat tincidunt et ac diam. Aliquam tincidunt augue sem, ut vestibulum est volutpat eget. Suspendisse potenti. Integer condimentum, risus nec maximus elementum, lacus purus porta arcu, at ultrices diam nisl eget urna. Curabitur sollicitudin diam quis sollicitudin varius. Ut porta erat ornare laoreet euismod. In tincidunt purus dui, nec egestas dui convallis non. In vestibulum ipsum in dictum scelerisque.

Mauris interdum porttitor fringilla. Proin tincidunt sodales leo at ornare. Donec tempus magna non mauris gravida luctus. Cras vitae arcu vitae mauris eleifend scelerisque. Nam sem sapien, vulputate nec felis eu, blandit convallis risus. Pellentesque sollicitudin venenatis tincidunt. In et ipsum libero. Nullam tempor ligula a massa convallis pellentesque. Mauris interdum porttitor fringilla. Proin tincidunt sodales leo at ornare. Donec tempus magna non mauris gravida luctus. Cras vitae arcu vitae mauris eleifend scelerisque. Nam sem sapien, vulputate nec felis eu, blandit convallis risus. Pellentesque sollicitudin venenatis tincidunt. In et ipsum libero. Nullam tempor ligula a massa convallis pellentesque.

%------------------------------------------------

\section{Results}

\begin{table} % Single column table
	\caption{Example single column table.}
	\centering
	\begin{tabular}{l l r}
		\toprule
		\multicolumn{2}{c}{Location} \\
		\cmidrule(r){1-2}
		East Distance & West Distance & Count \\
		\midrule
		100km & 200km & 422 \\
		350km & 1000km & 1833 \\
		600km & 1200km & 890 \\
		\bottomrule
	\end{tabular}
	\label{tab:distcounts}
\end{table}

Referencing a table using its label: Table \ref{tab:distcounts}.

\begin{table*} % Full width table (notice the starred environment)
	\caption{Example two column table with fixed-width columns.}
	\centering % Horizontally center the table
	\begin{tabular}{L{0.2\linewidth} L{0.2\linewidth} R{0.15\linewidth}} % Manually specify column alignments with L{}, R{} or C{} and widths as a fixed amount, usually as a proportion of \linewidth
		\toprule
		\multicolumn{2}{c}{Location} \\
		\cmidrule(r){1-2}
		East Distance & West Distance & Count \\
		\midrule
		100km & 200km & 422 \\
		350km & 1000km & 1833 \\
		600km & 1200km & 890 \\
		\bottomrule
	\end{tabular}
\end{table*}

Aenean feugiat pellentesque venenatis. Sed faucibus tristique tortor vel ultrices. Donec consequat tellus sapien. Nam bibendum urna mauris, eget sagittis justo gravida vel. Mauris nisi lacus, malesuada sit amet neque ut, venenatis tempor orci. Curabitur feugiat sagittis molestie. Duis euismod arcu vitae quam scelerisque facilisis. Praesent volutpat eleifend tortor, in malesuada dui egestas id. Donec finibus ac risus sed pellentesque. Donec malesuada non magna nec feugiat. Mauris eget nibh nec orci congue porttitor vitae eu erat. Sed commodo ipsum ipsum, in elementum neque gravida euismod. Cras mi lacus, pulvinar ut sapien ut, rutrum sagittis dui. Donec non est a metus varius finibus. Pellentesque rutrum pellentesque ligula, vitae accumsan nulla hendrerit ut.

\begin{figure} % Single column figure
	\includegraphics[width=\linewidth]{Tolmukapea.jpg}
	\caption{Anther of thale cress (Arabidopsis thaliana), fluorescence micrograph. Source: Heiti Paves, \href{https://commons.wikimedia.org/wiki/File:Tolmukapea.jpg}{https://commons.wiki-\\media.org/wiki/File:Tolmukapea.jpg}.}
	\label{fig:tcanther}
\end{figure}

Referencing a figure using its label: Figure \ref{fig:tcanther}.

Aenean porttitor eros non pharetra congue. Proin in odio in dolor luctus auctor ac et mi. Etiam euismod mi sed lectus fringilla pretium. Phasellus tristique maximus lectus et sodales. Mauris feugiat ligula quis semper luctus. Nam sit amet felis sed leo fermentum aliquet. Mauris arcu dui, posuere id sem eget, cursus pulvinar mi. Donec nec lacus non lectus fermentum scelerisque et at nibh. Sed tristique, metus ac vestibulum porta, tortor lectus placerat lorem, et convallis tellus dolor eget ante. Pellentesque dui ligula, hendrerit a purus et, volutpat tempor lectus. Mauris nec purus nec mauris rhoncus pellentesque. Quisque quis diam sed est lacinia congue. Donec magna est, hendrerit sed metus vel, accumsan rutrum nibh.

\begin{figure*} % Two column figure (notice the starred environment)
	\includegraphics[width=\linewidth]{Fibroblastid.jpg}
	\caption{Bovine pulmonary artery endothelial cells in culture. Blue: nuclei; red: mitochondria; green: microfilaments. Computer generated image from a 3D model based on a confocal laser scanning microscopy using fluorescent marker dyes. Source: Heiti Paves, \href{https://commons.wikimedia.org/wiki/File:Fibroblastid.jpg}{https://commons.wikimedia.org/wiki/File:Fibroblastid.jpg}.}
	\label{fig:bpartery}
\end{figure*}

Orci varius natoque penatibus et magnis dis parturient montes, nascetur ridiculus mus. Etiam cursus lectus purus, tempus iaculis quam dictum tristique. Nam interdum sapien nec tempor mattis. Quisque id sapien nisi. Mauris vehicula ornare eros vel efficitur. Nulla consectetur, turpis quis fringilla tincidunt, mi neque iaculis lectus, vel commodo elit odio non ex. Duis facilisis, purus ac viverra iaculis, turpis lectus ultrices ante, ac vestibulum ligula magna in libero. Etiam tristique maximus lacinia. Vestibulum hendrerit, lacus malesuada laoreet blandit, sapien velit sollicitudin nunc, eu porttitor urna ligula at lorem. Aliquam faucibus eros in fermentum venenatis. Fusce consectetur congue pellentesque. Suspendisse at nisi sit amet est porttitor cursus. Cras placerat faucibus nunc, a laoreet justo dignissim sit amet.

\subsection{International Support}

\noindent àáâäãåèéêëìíîïòóôöõøùúûüÿýñçčšž

\noindent ÀÁÂÄÃÅÈÉÊËÌÍÎÏÒÓÔÖÕØÙÚÛÜŸÝÑ

\noindent ßÇŒÆČŠŽ

\subsection{Links}

This is a clickable URL link: \href{https://www.latextemplates.com}{LaTeX Templates}. This is a clickable email link: \href{mailto:vel@latextemplates.com}{vel@latextemplates.com}. This is a clickable monospaced URL link: \url{https://www.LaTeXTemplates.com}.

%------------------------------------------------

\section{Discussion}

This statement requires citation \autocite{Smith:2023qr}. This statement requires multiple citations \autocite{Smith:2023qr, Smith:2024jd}. This statement contains an in-text citation, for directly referring to a citation like so: \textcite{Smith:2024jd}.

\subsection{Subsection One}

Suspendisse potenti. Vivamus suscipit dapibus metus. Proin auctor iaculis ex, id fermentum lectus dapibus tristique. Nullam maximus eros eget leo pretium dapibus. Nunc in auctor erat, id interdum risus. Suspendisse aliquet vehicula accumsan. In vestibulum efficitur dictum. Sed ultrices, libero nec fringilla feugiat, elit massa auctor ligula, vehicula tempor ligula felis in lectus. Suspendisse sem dui, pharetra ut sodales eu, suscipit sit amet felis. Donec pretium viverra ante, ac pulvinar eros. Suspendisse gravida consectetur urna. Pellentesque vitae leo porta, imperdiet eros eget, posuere sem. Praesent eget leo efficitur odio bibendum condimentum sit amet vel ex. Nunc maximus quam orci, quis pulvinar nibh eleifend ac. Quisque consequat lacus magna, eu posuere tellus iaculis ac. Sed vitae tortor tincidunt ante sagittis iaculis.

\subsection{Subsection Two}

Nullam mollis tellus lorem, sed congue ipsum euismod a. Donec pulvinar neque sed ligula ornare sodales. Nulla sagittis vel lectus nec laoreet. Nulla volutpat malesuada turpis at ultricies. Ut luctus velit odio, sagittis volutpat erat aliquet vel. Donec ac neque eget neque volutpat mollis. Vestibulum viverra ligula et sapien bibendum, vel vulputate ex euismod. Curabitur nec velit velit. Aliquam vulputate lorem elit, id tempus nisl finibus sit amet. Curabitur ex turpis, consequat at lectus id, imperdiet molestie augue. Curabitur eu eros molestie purus commodo hendrerit. Quisque auctor ipsum nec mauris malesuada, non fringilla nibh viverra. Quisque gravida, metus quis semper pulvinar, dolor nisl suscipit leo, vestibulum volutpat ante justo ultrices diam. Sed id facilisis turpis, et aliquet eros.

\subsubsection{Subsubsection Example}

Duis venenatis eget lectus a aliquet. Integer vulputate ante suscipit felis feugiat rutrum. Aliquam eget dolor eu augue elementum ornare. Nulla fringilla interdum volutpat. Sed tincidunt, neque quis imperdiet hendrerit, turpis sapien ornare justo, ac blandit felis sem quis diam. Proin luctus urna sit amet felis tincidunt, sed congue nunc pellentesque. Ut faucibus a magna faucibus finibus. Etiam id mi euismod, auctor nisi eget, pretium metus. Proin tincidunt interdum mi non interdum. Donec semper luctus dolor at elementum. Aenean eu congue tortor, sed hendrerit magna. Quisque a dolor ante. Mauris semper id urna id gravida. Vestibulum mi tortor, finibus eu felis in, vehicula aliquam mi.

Aliquam arcu turpis, ultrices sed luctus ac, vehicula id metus. Morbi eu feugiat velit, et tempus augue. Proin ac mattis tortor. Donec tincidunt, ante rhoncus luctus semper, arcu lorem lobortis justo, nec convallis ante quam quis lectus. Aenean tincidunt sodales massa, et hendrerit tellus mattis ac. Sed non pretium nibh. 

Donec cursus maximus luctus. Vivamus lobortis eros et massa porta porttitor. Nam vitae suscipit mi. Pellentesque ex tellus, iaculis vel libero at, cursus pretium sapien. Curabitur accumsan velit sit amet nulla lobortis, ut pretium ex aliquam. Proin eget volutpat orci. Morbi eu aliquet turpis. Vivamus molestie urna quis tempor tristique. Proin hendrerit sem nec tempor sollicitudin.

%----------------------------------------------------------------------------------------
%	 REFERENCES
%----------------------------------------------------------------------------------------

\printbibliography % Output the bibliography

%----------------------------------------------------------------------------------------

\end{document}
